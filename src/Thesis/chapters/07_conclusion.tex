% !TeX root = ../main.tex
\chapter{Fazit und Ausblick}\label{ch:conclusion}

Diese Arbeit hat sich mit Introspected REST, einem neuen Architekturstil, welcher auf REST aufbaut, beschäftigt. Introspected REST übernimmt die sechs REST-Prinzipien, konkretisiert das Prinzip \enquote{Hypermedia as the Engine of Application State} aber durch das \emph{Introspection-Prinzip} \enquote{Introspection as the Engine of Application State}. Dieses besagt, dass Nutzdaten und statische Metadaten, allen voran Hypermedia, getrennt werden sollen. Clients stellen also einen Request, um die Nutzdaten abzufragen, und einen zweiten, um an die Metadaten zu gelangen. Neben dem Introspection-Prinzip bilden \emph{Microtypes} einen weiteren Grundpfeiler von Introspected REST. Während die Mediatypes von REST-APIs meist monolithisch aufgebaut sind und die komplette Funktionalität der API beschreiben, soll der verwendete Mediatype einer Introspected-REST-API aus unabhängigen Bausteinen, den Microtypes, zusammengesetzt werden. Ein Microtype beschreibt nur einen kleinen Teil der API-Funktionalität, z.B., wie eine Liste von Elementen gefiltert wird oder wie Fehlermeldungen ausgegeben werden. Die verwendeten Microtypes werden zwischen Client und Server durch Content-Negotiation ausgehandelt. Dies ermöglicht einem Client, genauer als bei REST-APIs zu bestimmen, welche Daten der Server zurückgeben soll, und die gleiche API kann von einer Vielzahl von Clients verwendet werden. Eine weitere Erleichterung für Clients könnte durch die API-übergreifende Verwendung standardisierter Microtypes erreicht werden, denn so erhöht sich die Konsistenz zwischen verschiedenen APIs und die Wiederverwendung von Clientcode wird ermöglicht.

Introspected-REST wurde von Filippos Vasilakis, dem Erfinder von Introspected REST, als Alternative zu heutzutage verbreiteten API-Stilen positioniert. Um diese Behauptung zu untersuchen, wurde ein Vergleich von Introspected REST mit REST sowie GraphQL, einer Datenabfragesprache, und gRPC, einem Protokoll für effiziente Interprozesskommunikation, gezogen. Diese drei alternativen API-Stile werden heutzutage am meisten verwendet. Als Vergleichskriterien wurden die Performance, die Evolvierbarkeit sowie die Komplexität/Benutzbarkeit gewählt.

Die Performance wurde aus der Perspektive eines Clients bewertet. Dazu wurden APIs zur Verwaltung von Notizen mit den vier zu vergleichenden API-Stilen erstellt. Ein oder mehrere Clients mussten nun Anwendungsfälle durchspielen. Für jeden Anwendungsfall wurde die benötigte Zeit sowie die Anzahl ausgetauschter Bytes gemessen. Die Messergebnisse zeigten, dass die Introspected-REST- und REST-APIs sehr \enquote{geschwätzig} waren und viele Daten zwischen Client und Server ausgetauscht wurden, während GraphQL und gRPC mit einer kleineren Datenmenge zurechtkamen. Entsprechend benötigten die Introspected-REST- und REST-Clients auch mehr Zeit, um einen Anwendungsfall abzuschließen. Eine Verbesserung von Introspected REST gegenüber REST hinsichtlich der Performance konnte nicht festgestellt werden \textendash{} zumindest nicht bei einem Client, welcher Hypermedia anfordert.

Für den Vergleich der Evolvierbarkeit wurden veränderte Anforderungen an eine API simuliert. Für jeden API-Stil wurde beschrieben, ob eine Änderung abwärtskompatibel ist, d.h., ob der Client ohne Anpassungen weiterhin mit dem Server kommunizieren kann, und, falls nicht, warum Anpassungen erforderlich sind. Es zeigte sich, dass Introspected REST und REST mit Änderungen besser zurechtkommen als GraphQL und gRPC, da Businesslogik vom Server kontrolliert und nicht im Client dupliziert wird. Es konnten nur sehr geringe Unterschiede zwischen Introspected REST und REST bei dieser Untersuchung festgestellt werden. Durch Microtypes können Clients zwar einen Microtype leicht durch einen anderen ersetzen, doch beeinflusst dies nicht, wie gut ein individueller Microtype weiterentwickelt werden kann. Die Evolvierbarkeit wird durch Introspection dahingehend erhöht, dass APIs, welche zuvor Hypermedia nicht verwendet haben, dies nun tun können, ohne Änderungen an bestehenden Clients zu fordern.

Die Komplexität wurde anhand von Heuristiken für die Benutzbarkeit einer API bewertet. Für jeden API-Stil wurde festgestellt, wie gut dieser eine Heuristik realisiert. Es ergab sich dabei ein gemischtes Bild: Jeder API-Stil zeigte individuelle Stärken und Schwächen. Introspected REST blieb vor allem im Bereich der Fehlervermeidung zurück. GraphQL und gRPC ermöglichen z.B.\ durch einen expliziten Schnittstellenvertrag eine bessere Validierung der Eingaben von Benutzerinnen und Benutzern im Client. Dafür verbessern vor allem Microtypes die Konsistenz einer Introspected-REST-API im Vergleich zu REST.

\para{}Zusammenfassend lässt sich sagen, dass Introspected REST versucht, einen pragmatischen Mittelweg zu gehen zwischen einer geringeren Komplexität und trotzdem guter Evolvierbarkeit. Pragmatisch deshalb, weil in der Realität viele Clients von REST-APIs Hypermedia nicht verwenden und stattdessen URLs und Businesslogik hart kodieren. Eine Introspected-REST-API verbessert die Performance und Benutzbarkeit für solche Clients, da sie nicht mehr genötigt werden, mit Hypermedia umzugehen. Ebenfalls gibt Introspected REST durch die Idee der Microtypes den API-Entwicklerinnen und -Entwicklern ein Werkzeug an die Hand, um mit der Herausforderung vieler verschiedener Anforderungen durch eine Vielzahl von Clients umzugehen. Der einzige API-Stil, der sonst einen konkreten Ansatz in dieser Hinsicht bietet, ist GraphQL. Doch existieren drei Jahre nach der Präsentation von Introspected REST immer noch keine öffentlichen Microtypes. Um eine Adoption des API-Stils zu realisieren, wird aber eine kritische Masse dieser benötigt, sodass ein API-Anbieter nicht die Last des Designs aller benötigten Microtypes allein tragen muss. Bisher wird auch noch kein Prozess für die Standardisierung von Microtypes vorgeschlagen, sodass hier die Gefahr besteht, dass jeder sein eigenes Süppchen kocht \textendash{} sollte Introspected REST eine messbare Verbreitung erreichen. Das Microtype-Ökosystem ist jedenfalls noch in weiter Ferne.

Zwar dauerte es auch einige Jahre, bis REST als API-Stil Verwendung fand. Doch klang REST zu diesem Zeitpunkt wie eine willkommene Alternative zu SOAP, welches zu einer engen Kopplung zwischen Client und Server führte, XML verwendete und nicht gemacht war für das Web. Eine derartige Situation liegt heute nicht vor. API-Anbieter haben die Wahl zwischen REST für langlebige APIs, GraphQL für datenintensive, clientzentrierte APIs, gRPC für interne APIs und anderen HTTP-basierten APIs, die oft fälschlicherweise als \enquote{REST APIs} bezeichnet werden, falls sie den für sie bequemen und bekannten Ansatz wählen wollen. Und dann gibt es noch viele andere, weniger verbreitete API-Stile. Introspected REST muss sich also gegen eine Vielzahl von Konkurrenten durchsetzen. Wenn man als API-Anbieter nun verschiedene Gruppen von Entwicklerinnen und Entwicklern bedienen muss, vor allem solche, die einen Client, der langfristig wenige Anpassungen erfordert, bauen wollen, und solche, die einen HTTP-Client in der Weise erstellen möchten, wie sie es gewohnt sind, kann Introspected REST theoretisch ein guter Ansatz sein. Doch da bis jetzt keine Infrastruktur dafür vorhanden ist, gleichen die Vorteile von Introspected REST den initialen Aufwand wahrscheinlich nicht aus.

\para{}In dieser Arbeit wurden Introspected-REST-Clients als hypermediaaffin angenommen. Es bleibt offen, wie vorteilhaft der API-Stil für Clients ohne Hypermediaverwendung ist. Eine weitere interessante Frage ist, ob Introspected-REST- und REST-APIs mit feingliedriger Aufteilung der Ressourcen durch den Einsatz von HTTP/2 eine ähnlich gute Performance erreichen können wie GraphQL. Die Beschreibung von Introspected REST durch Vasilakis empfand der Autor als nicht sehr detailreich und offen für Interpretationen, die sich in der Implementierung widerspiegeln. Es ist wünschenswert, dass diese Interpretationsspielräume geschlossen oder konkurrierende Interpretationen empirisch untersucht werden. Eine Lehre aus Introspected REST ist, dass Mediatypes in der jetzigen Form nicht ideal sind und darüber nachgedacht werden sollte, wie man sie änderbarer gestaltet. Microtypes sind ein möglicher Weg, doch könnten während der Erforschung des Problems sicherlich alternative oder sogar bessere Ansätze gefunden werden.

\para{}Es bleibt also weiterhin Bewegung im Web-API-Bereich und auch Introspected REST bildet nicht das Ende der Geschichte.