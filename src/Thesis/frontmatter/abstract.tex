\thispagestyle{plain}
\chapter*{Kurzzusammenfassung}

Introspected REST ist ein neuer Ansatz für die Entwicklung von Web-APIs, welcher auf REST aufbaut. In der vorliegenden Arbeit wird dieser neue API-Stil mit REST, GraphQL und gRPC hinsichtlich der Performance, Evolvierbarkeit und Komplexität bzw.\ Benutzbarkeit verglichen. Die Performance der untersuchten Introspected-REST-API liegt dabei im gleichen Bereich wie die der REST-API, welche ihrerseits von gRPC und GraphQL übertroffen wird. Auch die Evolvierbarkeit ist gleich gut im Vergleich zu REST. Die Verwendung von Introspected REST trägt vor allem zu einer besseren Benutzbarkeit der API bei.
\bigskip{}

{\let\clearpage\relax\par \chapter*{Abstract}}

Introspected REST is a new approach to the development of web APIs. It builds upon the REST architectural style. In this thesis, Introspected REST is compared to REST, GraphQL, and gRPC in terms of performance, evolvability, and complexity/usability. The results show that the performance of Introspected REST is in the same order of magnitude as the performance of REST\@. Both are in turn outperformed by gRPC and GraphQL, respectively. The evolvability rates similarly to REST's evolvability, too. Using Introspected REST for an API does most notably improve its usability.